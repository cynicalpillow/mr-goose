\documentclass[11pt]{article}
\usepackage{amsmath}
\usepackage{amssymb}
\usepackage{fullpage}
\usepackage{hyperref}
\setlength\parindent{0pt}
\title{Lecture 2 (1.2)}
\author{Rui Li}


\begin{document}
\maketitle
\tableofcontents
\section*{1.2 Introduction to Sequences}
\addcontentsline{toc}{section}{1.2 Introduction to Sequences}
\textbf{Sequence: }an ordered list of numbers:
1, 3, 4, 5, 7, 0, -1, 20001 (finite!)
\medskip

An infinite sequence: \{2, 4, 8, 16..., $2^n$,...\} or \{$2^n$\}.

Notation: $\{a_1, a_2, a_3,...,a_n,....\}$ or $\{a_n\}$from $n=1$ to infinite, or simply $\{a_n\}$.
\bigskip

\textbf{Definition: } Explicitly (like above), or recursively:
$a_1 = 2, a_n+1 = 2\times a_n$ (n is called the index, $a_n$ are called elements or terms)

\subsection*{\underline{Subsequences and Tails}}
\addcontentsline{toc}{subsection}{Subsequences and Tails}

\textbf{Definition: }

Let $\{a_n\}$ be a sequence. Let $\{n_1, n_2, n_3,...,n_k,...\}$ be a sequence of natural numbers such that $n_1 < n_2 < n_3 < ... < n_k < ...$ A subsequence of $\{a_n\}$ is a sequence of the form $\{a_{n_k}\} = \{a_{n_1}, a_{n_2}, a_{n_3},...,a_{n_k},...\}$. $\{n1, n2, n3,...,nk,...\}$ represents the indexes of the subsequence taken from the original \{an\} sequence.
\bigskip

A tail of a sequence is when given a sequence $\{a_n\}$ and $k \in \mathbb{N}$, the subsequence $\{a_k, a_{k+1}, a_{k+2},...\}$ is called the tail of $\{a_n\}$ with a cutoff k.

\subsection*{\underline{Limits of Sequences}}
\addcontentsline{toc}{subsection}{Limits of Sequences}

\textbf{Formal definition of the Limit of a Sequence:}

We say that L is the limit of the sequence $\{a_n\}$ as n goes to infinite if for every $\epsilon > 0$, there exists an $N \in \mathbb{N}$ such that if $n \geq N$ then

\begin{center}
    $|an - L| < \epsilon$
\end{center}

If such an L exists, we say that the sequence is convergent and write:

\begin{center}
    $$\lim_{n\to\infty} a_n = L$$
\end{center}

We may also use the notation $a_n \rightarrow L$ to mean $\{a_n\}$ converges to L.

If no such L exists, then we say that the sequence diverges.

\bigskip

\textbf{Examples: }

\textit{\underline{1) Prove $a_n=\frac{1}{n^2}$, L = 0}}
\smallskip

If $\epsilon = \frac{1}{100}$, will there be some N for which $n \geq N$ and $|
\frac{1}{n^2} - 0| < \frac{1}{100}$?
\medskip

We can rearrange like this: $\frac{1}{n^2} < \frac{1}{100} \rightarrow 100 < n^2$

If $n > 10$ then, $N = 10, n \geq 10 \rightarrow |\frac{1}{n^2} - 0| < \frac{1}{100}$

Since $n \geq N$ and $|\frac{1}{n^2} - 0| < \epsilon$, we can write this as:

$\frac{1}{n^2} < \epsilon \rightarrow n^2 > \frac{1}{\epsilon} \rightarrow n > \frac{1}{\sqrt{\epsilon}}$

$N \geq \frac{1}{\sqrt{\epsilon}} \rightarrow |\frac{1}{n^2} - 0| < \epsilon$
\medskip

Essentially, what we've proven here is that if we rearrange n so that we get $n > \frac{1}{\sqrt{\epsilon}}$, then we know that if N is greater or equal to $\frac{1}{\sqrt{\epsilon}}$, then this inequality will be true. So when $N > \frac{1}{\sqrt{\epsilon}}$ then we know that $n^2 > N^2 \geq \frac{1}{\epsilon}$. 

Given this fact, we can prove that if we rearrange that inequality, we get $\frac{1}{n^2} < \epsilon$, meaning $|\frac{1}{n^2} - 0| < \epsilon$ if $N > \frac{1}{\sqrt{\epsilon}}$. Essentially, we are rearranging the inequality so we find what N has to be for the formal definition of the limit to be true, and  returning to the original statement.
\medskip

\textit{\underline{2) Prove $\lim_{n\to\infty} \frac{n}{2n+3} = \frac{1}{2}$}}
\smallskip

We know $\epsilon > 0, n > N$.

Step by step:

\begin{align}
    |\frac{n}{2n+3} - \frac{1}{2}| &< \epsilon\\
    |\frac{2n-2n-3}{4n+6}| &< \epsilon\\
\end{align}

\end{document}