\documentclass[11pt]{article}
\usepackage{amsmath}
\usepackage{amssymb}
\usepackage{fullpage}
\usepackage{hyperref}
\setlength\parindent{0pt}
\title{Lecture 3 (1.4-2.3)}
\author{Rui Li}


\begin{document}
\maketitle
\tableofcontents
\section*{1.4.3 Negation of Quantified statements}
\addcontentsline{toc}{section}{1.4.3 Negation of Quantified statements}
Consider: "Everyone in this room is right handed"

Which in symbols is $\forall x \in$ S, P(x)
\bigskip

S = set of people in room

P(x) is "x is right-handed"

The negation is "Someone in the room is not right-handed"
$\exists x \in S$, $\neg P(x)$

Thus, $\neg(\forall x \in S, P(x)) = \exists x \in S, \neg P(x)$

Also since $\neg(\neg P) = P$

$\neg(\exists x \in S), P(x)) = \forall x \in S, \neg P(x)$
\bigskip

Negation of $x^2 - x >= 0$ for all real numbers x (false) is 

$x^2 -x < 0$ for some real numbers x (true)

$\neg(\forall x \in \mathbb{R})$

\end{document}