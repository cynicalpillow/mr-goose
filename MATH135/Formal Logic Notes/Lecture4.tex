\documentclass[11pt]{article}
\usepackage{amsmath}
\usepackage{amssymb}
\usepackage{fullpage}
\usepackage{hyperref}
\setlength\parindent{0pt}
\title{Lecture 4 (2.4 - 2.6)}
\author{Rui Li}


\begin{document}
\maketitle
\tableofcontents

\bigskip

Example: 

Show that:

\begin{align*}
    \neg(P \wedge (\neg Q \wedge R)) \equiv \neg(P \wedge Q) \vee R
\end{align*}

Solution:

\begin{align*}
    \neg(P \wedge (\neg Q \wedge R)) &\equiv \neg P \vee \neg(\neg Q \wedge R)\ (By\ DML)\\
    &\equiv \neg P \vee \neg \neg Q \vee \neg R\\
    &\equiv \neg P \vee (Q \vee \neg R)\\
    &\equiv \neg P \vee (\neg R \vee Q)\ (By\ Commutative\ Law)\\
    &\equiv (\neg P \vee \neg R) \vee Q)\ (By\ Associative\ Law)\\
    &\equiv \neg(P \wedge R) \vee Q\ (By\ DML)
\end{align*}

\section*{2.4.1 Implication}
\addcontentsline{toc}{section}{2.4.1 Implication}

\textbf{Definition: }
\begin{align*}
    &P \rightarrow Q\\
    (P\ &implies\ Q)\\
    (If&\ P\ then\ Q)
\end{align*}

\begin{center}
    \begin{tabular}{ |c|c|c| } 
        \hline
        P & Q & $P \rightarrow Q$ \\ 
        \hline
        True & True & True \\ 
        True & False & False \\
        False & True & True \\
        False & False & True \\ 
        \hline
    \end{tabular}
\end{center}

$P \rightarrow Q$ is called a conditional statement, where P is the hypothesis and Q is the conclusion.

\bigskip

Examples: 

\medskip

P = "You Study", Q = "You pass", $P \rightarrow Q = $ "If you study then you pass"

\medskip

For all real x, if $x > 2$ then $x^2 > 4$

For all real x, if $x \geq 2$ then $x^2 > 4$

\bigskip

\textbf{Negating Implications}

\begin{center}
    \begin{tabular}{ |c|c|c|c| } 
        \hline
        P & Q & $P \rightarrow Q$ & $\neg P \vee Q$ \\ 
        \hline
        True & True & True & True\\ 
        True & False & False & False\\
        False & True & True & True\\
        False & False & True & True\\ 
        \hline
    \end{tabular}
\end{center}

\begin{align*}
    P \rightarrow Q \equiv \neg P \vee Q
\end{align*}

Thus, 
\begin{align*}
    \neg(P \rightarrow Q) &\equiv \neg(\neg P \vee Q)\\
    &\equiv \neg \neg P \wedge \neg Q\\
    &\equiv P \wedge \neg Q
\end{align*}

\section*{2.5 Converse and Contrapositive}
\addcontentsline{toc}{section}{2.5 Converse and Contrapositive}

\textbf{Definition: }

\textbf{Converse of $P \rightarrow Q$ is $Q \rightarrow P$}

\textbf{Contrapositive of $P \rightarrow Q$ is $(\neg Q) \rightarrow (\neg P)$}

However, $P \rightarrow Q \neq Q \rightarrow P$

Additionally $\neg Q \rightarrow \neg P \equiv P \rightarrow Q$

\medskip
Example:

"If $a < b$ then $a \leq b$"

Converse: "If $a \leq b$ then $a < b$"

Contrapositive: "If $a > b$ then $a \geq b$"

\section*{2.6 If and Only If}
\addcontentsline{toc}{section}{2.6 If and Only If}

\textbf{Definition: }

$P \leftrightarrow Q$ (P if and only if Q)

\medskip

\begin{center}
    \begin{tabular}{ |c|c|c| } 
        \hline
        P & Q & $P \leftrightarrow Q$ \\ 
        \hline
        True & True & True \\ 
        True & False & False \\
        False & True & False \\
        False & False & False \\ 
        \hline
    \end{tabular}
\end{center}

\end{document}