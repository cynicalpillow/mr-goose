\documentclass[11pt]{article}
\usepackage{amsmath}
\usepackage{amssymb}
\usepackage{fullpage}
\usepackage{hyperref}
\setlength\parindent{0pt}
\title{Lecture 3 (1.4 - 2.3)}
\author{Rui Li}


\begin{document}
\maketitle
\tableofcontents
\section*{1.4.3 Negation of Quantified statements}
\addcontentsline{toc}{section}{1.4.3 Negation of Quantified statements}
Consider: "Everyone in this room is right handed"

Which in symbols is, 
\begin{align*}
    \forall x \in S, P(x)
\end{align*}

\bigskip

\textbf{S = set of people in room}

\textbf{P(x) is "x is right-handed"}

The negation is "Someone in the room is not right-handed",
\begin{align*}
    \exists x \in S, \neg P(x)
\end{align*}

Thus, 
\begin{align*}
    \neg(\forall x \in S, P(x)) = \exists x \in S, \neg P(x)
\end{align*}

Also since, 
\begin{align*}
    \neg(\neg P) &= P\\
    \neg(\exists x \in S), P(x)) &= \forall x \in S, \neg P(x)
\end{align*}
\medskip

Negation of $x^2 - x >= 0$ for all real numbers x (false) is, 
\begin{align*}
    x^2 -x < 0\ for\ some\ real\ numbers\ x\ (true)\\
    \neg(\forall x \in \mathbb{R}, x^2-x \geq 0)\equiv(\exists x \in \mathbb{R}, x^2 - x < 0)
\end{align*}

\section*{1.5 Nested Quantifiers}
\addcontentsline{toc}{section}{1.5 Nested Quantifiers}
Examples:

a) $\exists m \in \mathbb{Z}, \exists n \in \mathbb{Z}, m+n\ is\ even\ (True)$

b) $\forall m \in \mathbb{Z}, \forall n \in \mathbb{Z}, m+n\ is\ odd\ (False)$

c) $\exists m \in \mathbb{Z}, \forall n \in \mathbb{Z}, m+n\ is\ odd\ (False)$

d) $\forall m \in \mathbb{Z}, \exists n \in \mathbb{Z}, m+n\ is\ even\ (True)$

e) $\exists m \in \mathbb{Z}, \forall n \in \mathbb{Z}, m+n\ is\ even\ (False)$

f) $\forall m \in \mathbb{Z}, \exists n \in \mathbb{Z}, m+n\ is\ odd\ (True)$

Variables to the right depend on variables to the left.

From above, A and B are negations of each other,

\begin{align*}
    \neg(\exists m \in \mathbb{Z}, \exists n \in \mathbb{Z}, m+n\ is\ even) \equiv \forall m \in \mathbb{Z}, \forall n \in \mathbb{Z}, m+n\ is\ odd\ (not\ even)
\end{align*}

C and D and E and F are negations as well

In general \textbf{flip quantifiers, and negate the open sentence P(x).}

The order of quantifiers matters in statements with mixed quantifiers.

\section*{2.2 Conjunctions and Disjunctions}
\addcontentsline{toc}{section}{2.2 Conjunctions and Disjunctions}

\textbf{Definition: }Conjunction (P AND Q) "$P \wedge Q$" is defined as,

\begin{center}
    \begin{tabular}{ |c|c|c| } 
        \hline
        P & Q & $P \wedge Q$ \\ 
        \hline
        True & True & True \\ 
        True & False & False \\
        False & True & False \\
        False & False & False \\ 
        \hline
    \end{tabular}
\end{center}

Example:

\begin{align*}
    (1+1 = 3)\wedge(\pi + 2 < 6)
\end{align*}

is false because $1+1=3$ is false. However, 

\begin{align*}
    (1+1 = 2)\wedge(\pi + 2 < 6)
\end{align*}

is true.
\bigskip

\textbf{Definition: } Disjunction (P or Q) "$P \vee Q$" is defined as,

\begin{center}
    \begin{tabular}{ |c|c|c| } 
        \hline
        P & Q & $P \vee Q$ \\ 
        \hline
        True & True & True \\ 
        True & False & True \\
        False & True & True \\
        False & False & False \\ 
        \hline
    \end{tabular}
\end{center}

$\neg P$, $P \wedge Q$, $P \vee Q$, are examples of compound statements, ie. statements that are composed of one or more \underline{component statements} (P, Q, ...) via logical operators. ($\neg,\ \wedge, \vee,\ ...$)

\begin{center}
    \begin{tabular}{ |c|c|c|c|c|c|c| } 
        \hline
        P & Q & $(P \wedge Q)$ & $\neg(P \wedge Q)$ & $\neg P$ & $\neg Q$ & $(\neg P)\vee(\neg Q)$\\ 
        \hline
        True & True & True & False & False & False & False \\ 
        True & False & False & True & False & True & True \\ 
        False & True & False & True & True & False & True \\ 
        True & True & False & True & True & True & True \\ 
        \hline
    \end{tabular}
\end{center}

As $\neg(P \wedge Q)$ always has the same truth values as $(\neg P) \vee (\neg Q)$, they are logically equivalent.

\begin{align*}
    \neg(P \wedge Q) \equiv (\neg P) \vee (\neg Q)
\end{align*}

\section*{2.3 Logical Operators and Algebra}
\addcontentsline{toc}{section}{2.3 Logical Operators and Algebra}

\textbf{De Morgan's Laws (DML)}

1. $\neg(P \wedge Q) \equiv (\neg P) \vee (\neg Q)$

2. $\neg(P \vee Q) \equiv (\neg P) \wedge (\neg Q)$

\medskip

Example: The negation of

\textbf{Both $e + \pi$ and $e\pi$ are rational}

is

\textbf{Either $e + \pi$ is irrational or $e\pi$ is irrational}

\medskip

Practice:

Let P, Q, R be statements

\begin{align*}
    P \wedge (Q \vee R) \equiv (P \wedge Q) \vee (P \wedge R)-
\end{align*}

\begin{center}
    \begin{tabular}{ |c|c|c|c|c|c|c|c| } 
        \hline
        P & Q & R & $Q \wedge R$ & $P \wedge (Q \wedge R)$ & $P \wedge Q$ & $P \wedge R$ & $(P \wedge Q) \vee (P \wedge R)$\\ 
        \hline
        True & True & True & True & True & True & True & True \\ 
        True & True & False & True & True & True & False & True\\ 
        True & False & True & True & True & False & True & True\\ 
        False & True & True & True & False & False & False & False\\
        False & False & True & True & False & False & False & False\\ 
        False & True & False & True & False & False & False & False\\ 
        True & False& False & False & False & False & False & False\\ 
        False & False & False & False & False & False & False & False\\  
        \hline
    \end{tabular}
\end{center}

The truth table shows that, for every possible assignment of truth values P, Q, R, statement 1 has the same truth value as statement 2. Hence the two statements are logically equivalent.

\end{document}