\documentclass[11pt]{article}
\usepackage{amsmath}
\usepackage{amssymb}
\usepackage{fullpage}
\usepackage{hyperref}
\setlength\parindent{0pt} %No auto indents
\title{Lecture 2 (1.2 - 1.4)}
\author{Rui Li}

\begin{document}
\maketitle
\tableofcontents
\section*{1.2 - Sets}
\addcontentsline{toc}{section}{1.2 - Sets}
    \textbf{Definition: }a set is a collection of objects, called it's elements.
    \smallskip

    \textit{Notation: }
    
    $a \in S$ (a is an element of S)
    
    $a \notin S$ (a is not an element of S)
    \bigskip

    \underline{\textbf{The Natural Numbers}}
    
    $\mathbb{N}$ = \{1, 2, 3..\}
    
    Thus $1 \in \mathbb{N}$, $2 \in \mathbb{N}$, etc. However $0 \notin \mathbb{N}$. (in MATH135)
    \bigskip

    \underline{\textbf{The Integers}}
    
    $\mathbb{Z}$ = \{..-2, -1, 0, 1, 2..\}
    
    Thus $0 \in \mathbb{Z}$, $\pm1 \in \mathbb{Z}$, $\frac{1}{2} \notin \mathbb{Z}$.
    \bigskip

    \underline{\textbf{The Rational Numbers}}

    $\mathbb{Q}$ = \{$\frac{a}{b}$: a, b $\in \mathbb{Z}$, b$\neq 0$\}

    Thus $1 = \frac{1}{1} \in \mathbb{Q}; \frac{1}{2} \in \mathbb{Q}; \sqrt{2} \notin \mathbb{Q}$ (Lecture 9).
    \bigskip

    \underline{\textbf{The Real Numbers}}

    $\mathbb{R} \rightarrow \sqrt{2} \in \mathbb{R}; \sqrt{-1} \notin \mathbb{R}$ (Lecture 9).
    \bigskip

    \textit{Examples:}
    
    Odd natural numbers less than 10 is \{1, 3, 5, 7, 9\}.

    The set S = \{1, 2, \{3\}\} has three elements $1 \in S, 2 \in S, \{3\} \in S$ but $3 \notin S$.

    \{1, 2, 3\} = \{1, 3, 2\} = \{1, 2, 2, 3\}
    \bigskip

    \textbf{Order is not relevant with sets, neither is multiplicity (the second 2 above is redundant)}
    \medskip

    \underline{\textbf{The Empty Set}}

    $\varnothing = \{\}$ No elements.
    \medskip

    \textit{Examples:}

    Set of integers both even and odd is $\varnothing$.

    The set of elements common to both \{3\}, \{\{3\}\} is $\varnothing$.

    \{$\varnothing$\} has one element, $\varnothing$ ($\varnothing \in \{\varnothing\}, \varnothing \notin \varnothing$).

    $\{\varnothing, \{\varnothing\}\}$ has 2 elements.

\section*{1.3 - Statements and Negation}
\addcontentsline{toc}{section}{1.3 - Statements and Negation}

    \textbf{Definition: } A statement is a sentence that is either true or false.

    \textit{Examples of statements: }
    
    $1+1=2\\1+1=3\\\sqrt{2} \notin \mathbb{Q}$
    \medskip

    \textit{Not statements:}

    $x^2-x=0\\\frac{m-7}{2m+4}=5$
    \bigskip

    \underline{\textbf{Negation}}

    \textbf{Definition: }Let p be a statement, the negation of p is denoted $\neg$P, and is the statement with the opposite truth value.

    \begin{center}
        \begin{tabular}{ |c|c|c| } 
            \hline
            P & $\neg P$ & $\neg(\neg P)$ \\ 
            \hline
            True & False & True \\ 
            False & True & False \\ 
            \hline
        \end{tabular}
    \end{center}
    
    Double negation has the same truth values as P, so P is logically equivalent to it's double negation.
    
    \begin{center}
        \underline{$P\equiv\neg (\neg P)$}
    \end{center}
    \medskip

    Negation of 1+1=2 is $\neg(1+1=2)$, or $1+1\neq2$.

    Negation of $\sqrt{2} \in \mathbb{Q}$ is $\sqrt{2} \notin \mathbb{Q}$. $\neg(\sqrt{2} \notin \mathbb{Q})$ is $\sqrt{2} \in \mathbb{Q}$ so $\neg(\neg(\sqrt{2} \in \mathbb{Q}))$ is $\sqrt{2} \in \mathbb{Q}$.

\section*{1.4 - Quantifiers}
\addcontentsline{toc}{section}{1.4 - Quantifiers}



\end{document}