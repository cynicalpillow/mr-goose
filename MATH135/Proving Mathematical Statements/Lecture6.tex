\documentclass[11pt]{article}
\usepackage{amsmath}
\usepackage{amssymb}
\usepackage{fullpage}
\usepackage{hyperref}
\setlength\parindent{0pt}
\title{Lecture 6 (3.2 - 3.3)}
\author{Rui Li}


\begin{document}
\maketitle
\tableofcontents

\section*{3.2 Proving existentially quantified statements}
\addcontentsline{toc}{section}{3.2 Proving existentially quantified statements}

$\exists x \in S, P(x)$

Prove that there exists $m \in \mathbb{Z}$ such that,

\begin{align*}
    \frac{m-7}{2m+4} = 5
\end{align*}

Let $m = -3$ then $m \in \mathbb{Z}$, and,

\begin{align*}
    \frac{m-7}{2m+4} &= \frac{-3-7}{2(-3)+4}\\
    &= \frac{-10}{-2}\\
    &= 5
\end{align*}

Essentially, for proving existentially quantified statements, give one example.

\bigskip

\textbf{Disproving existentially quantified statements}

\medskip

For disproving existentially quantified statements, prove the universally quantified statement, $\forall x \in S, \neg P(x)$.

\medskip

Disprove: There exists $x \in \mathbb{R}$ such that $cos(2x) + sin(2x) = 3$.

\underline{Proof:} Let $x \in \mathbb{R}$ then,

\begin{align*}
    -1 \leq cos(2x) \leq 1\ and\ -1 \leq sin(2x) \leq 1
\end{align*}

Hence,

\begin{align*}
    cos(2x) + sin(2x) \leq 1+1 = 2 < 3
\end{align*}

There exists no x such that the statement could be true.

\section*{3.3 Proving implications}
\addcontentsline{toc}{section}{3.3 Proving implications}

Assume hypothesis is true, and try and get to conclusion.

\textbf{DO NOT ASSUME CONCLUSION}

\medskip

Example:



\end{document}