\documentclass[11pt]{article}
\usepackage{amsmath}
\usepackage{amssymb}
\usepackage{fullpage}
\usepackage{hyperref}
\setlength\parindent{0pt}
\title{Lecture 5 (3.1)}
\author{Rui Li}


\begin{document}
\maketitle
\tableofcontents

\section*{3.1 Proving universally quantified statements}
\addcontentsline{toc}{section}{3.1 Proving universally quantified statements}

$\forall x \in \mathbb{Z}, P(x)$

\medskip

\textbf{Choose a representative x of S...}

"Let x be an arbitrary element of S"

"Let $x \in S$"..."

\medskip

\textbf{Remember, examples prove nothing!}

\medskip

\textbf{Do not assume what you want to prove}

\medskip

\underline{Tips:}

\textbf{- For inequalities, case by case analysis is a good technique.}

\textbf{- Direct proof can be hard, write down what we know already, plus axioms and other known facts like trig identities.}

\textbf{- Don't include "discovery" part in proof}

\textbf{- It could be beneficial to write out several different equations to use throughout the proof, labeled with (1), (2), (3)... etc.}

\end{document}